%%%
\section{The low Mach number flow equations}
\newcommand{\etal}{{\it et al.}}

\pelelm\ solves the reacting Navier-Stokes flow equations in the \emph{low Mach number} regime~\cite{DayBell:2000,rehm1978equations,Majda:1985}. In the low Mach number regime, the characteristic fluid velocity is small compared to the sound speed, and the effect of acoustic wave propagation is unimportant to the overall dynamics of the system. As such, acoustic wave propagation is mathematically removed from the equations of motion, allowing for a time step based on an advective CFL condition and leading to an increase in the allowable time step of order $1/M$ over an explicit, fully compressible method ($M$ is the Mach number). In this framework, the total pressure is decomposed into the sum of a constant (ambient) thermodynamic pressure $P_0$ and a perturbational pressure $\pi$ such that $\pi/P_0 = \mathcal{O} (M^2)$. 

The considered set of equations is a system of PDEs with Advection-Diffusion-Reaction (ADR) processes constrained by an equation of state in the form of a divergence constraint on the velocity. The evolution equations for the velocity and thermodynamic variables reads:
\begin{eqnarray}
&&\frac{\partial (\rho \boldsymbol{u})}{\partial t} + 
\nabla \cdot \left(\rho  \boldsymbol{u} \boldsymbol{u} + \tau \right)
= -\nabla \pi + \rho \, \boldsymbol{F}  ,
\nonumber
\\
&&\frac{\partial (\rho Y_m)}{\partial t} +
\nabla \cdot \left( \rho Y_m \boldsymbol{u} + \boldsymbol{\mathcal{F}}_{m} \right)
= \rho \, \dot{\omega}_m,
\label{eq:gen}
\\
&&\frac{ \partial (\rho h)}{ \partial t} +
\nabla \cdot \left( \rho h \boldsymbol{u} + \boldsymbol{\mathcal{Q}} \right) = 0 ,
\nonumber
\end{eqnarray}
where $\rho$ is the density, $\boldsymbol{u}$ is the velocity, $h$ is the mass-weighted enthalpy, $T$ is temperature and $Y_m$ is the mass fraction of species $m$. $\dot{\omega}_m$ is the molar production rate for species $m$, the modeling of which will be described in Section~\ref{ChemKinetics}. $\tau$ is the stress tensor, $\boldsymbol{\mathcal{Q}}$ is the heat flux vector and $\boldsymbol{\mathcal{F}}_m$
%$\,=\,$$- \rho \boldsymbol{\mathcal{D}}_i \nabla X_{i}$ 
are the species diffusion vectors. These transport fluxes require the evaluation of transport coefficients (e.g., the viscosity $\mu$, the conductivity $\lambda$ and the diffusivity matrix $D$) which are computed using the library EGLIB~\cite{EGLIB}, as will be described in more depth in Section~\ref{DifFluxesEGLIB}. The momentum source term, $\boldsymbol{F}$, is a long-wavelength forcing term designed to establish and maintain turbulence with the desired properties. Note that the definition of enthalpy includes the standard enthalpy of formation, so there is no net change to $h$ due to reactions.

These evolution equations are supplemented by an equation of state for the thermodynamic pressure:
\begin{eqnarray}
P_0(\rho,Y_m,T)=\rho \mathcal{R} T \sum_m \frac{Y_m}{W_m}
\label{eq:eos}
\end{eqnarray}
where $W_m$ is the species $m$ molecular weight, and by a relationship between enthalpy, species and temperature:
\begin{eqnarray}
h=\sum_m Y_m h_m(T)
\end{eqnarray}
where $h_m$ is the species $m$ enthalpy. 

Neither species diffusion nor reactions redistribute the total mass, hence we have $\sum_m \boldsymbol{\mathcal{F}}_m = 0$ and $\sum_m \dot{\omega}_m = 0$. Thus, summing the species equations and using the definition $\sum_m Y_m \equiv 1$ we obtain the continuity equation:
\begin{eqnarray}
\frac{\partial \rho}{\partial t} + \nabla \cdot \rho \boldsymbol{u} = 0
\label{eq:cont}
\end{eqnarray}

Equations~(\ref{eq:gen}), together with the equation of state, (\ref{eq:eos}) form a differential-algebraic equation (DAE) system that describes evolution subject to a constraint.  A standard approach to attacking such a system is to differente the constraint until it can be recast as an initial value problem.  Following this procedure, we set the thermodynamic pressure constant in the frame of the fluid,
\begin{eqnarray}
\frac{DP_0}{Dt} = 0
\label{eq:deos}
\end{eqnarray}
and observe that if the initial conditions satisfy the constraint, an evolution satisfying (\ref{eq:deos}) 
will continue to satisfy the constraint over all time.  Expanding (\ref{eq:deos}) via the chain rule, and using
Eq.~\ref{eq:cont}:
\begin{eqnarray}
\nabla \cdot \boldsymbol{u} = \frac{1}{T}\frac{DT}{Dt} + W \sum_m \frac{1}{W_m} \frac{DY_m}{Dt} = S
\label{eq:veloconstr}
\end{eqnarray}
The constraint on the flow is apparent ($W$ here is the mean mixture molecular weight).





%%%
\subsection{Transport fluxes}
\label{sub:DifFluxes}
Expressions for the transport fluxes appearing in Eqs.~(\ref{eq:gen}) can be approximated in the Enskog-Chapman expansion as~\cite{Ern:1994multicomponent}:
 \begin{eqnarray*}
&&\boldsymbol{\mathcal{F}}_{m} = \rho Y_m \boldsymbol{V_m}
\\ [2mm]
&&\tau_{i,j} = - \Big(\kappa - \frac{2}{3} \mu \Big) \delta_{i,j} \frac{\partial {u_k}}{\partial x_k} - \mu \Big(\frac{u_i}{\partial x_j} + \frac{u_j}{\partial x_i}\Big)
\\ [2mm]
&&\boldsymbol{\mathcal{Q}} =  \sum_m h_m \boldsymbol{\mathcal{F}}_{m}  - \lambda' \nabla T - P_0 \sum_m \theta_m \boldsymbol{d_m}
\end{eqnarray*}
where $\mu$ is the shear viscosity, $\kappa$ is the bulk viscosity, and $\lambda'$ is the partial thermal conductivity. In the \textit{full matrix diffusion model}, the $m$ species diffusion velocities vectors $\boldsymbol{V_m}$ are given by:
 \begin{eqnarray*}
\boldsymbol{V_m} = - \sum_j  {D}_{m,j} \boldsymbol{d_j} - \theta_m \nabla ln(T)
\end{eqnarray*}
where ${D}_{m,j}$ are the binary diffusion coefficients, and $\theta_m$ are thermal diffusion coefficients associated with the Soret (mass concentration flux due to an energy gradient) and Dufour (the energy flux due to a mass concentration gradient) transport of species $m$. The transport driving force due to composition gradients, $\boldsymbol{d_m}$, is given by~\cite{Ern:1994multicomponent}:
 \begin{eqnarray*}
\boldsymbol{d_m} = \nabla X_m + (X_m -Y_m) \frac{\nabla P_0}{P_0}
\label{dmeqs}
\end{eqnarray*}

%%%
\section{The \pelelm\ equation set}
\label{SumUpEq}
For \pelelm, we make the following simplifying assumptions:
\begin{enumerate}
\item The bulk viscosity, $\kappa$ is ignored
\item The low Mach limit implies that there are no spacial gradients in the thermodynamic pressure
\item The \textit{mixture-averaged}\ diffusion model is assumed, and we ignore Dufour and Soret effects
\end{enumerate}

With these assumptions, the conservation equations take the following form:
\begin{eqnarray}
&&\frac{\partial (\rho \boldsymbol{u})}{\partial t} + 
\nabla \cdot \left(\rho  \boldsymbol{u} \boldsymbol{u} + \tau \right)
= -\nabla \pi + \rho \, \boldsymbol{F}  ,
\nonumber
\\
&&\frac{\partial (\rho Y_i)}{\partial t} +
\nabla \cdot \left( \rho Y_i \boldsymbol{u} + \boldsymbol{\mathcal{F}}_{i} \right)
= \rho \, \dot{\omega}_i,
\label{eq:pelelm}
\\
&&\frac{ \partial (\rho h)}{ \partial t} +
\nabla \cdot \left( \rho h \boldsymbol{u} + \boldsymbol{\mathcal{Q}} \right) = 0 ,
\nonumber
\end{eqnarray}
with
 \begin{eqnarray*}
&&\boldsymbol{\mathcal{F}}_{m} = \rho Y_m \boldsymbol{V_m} = - \rho D_{m,mix} \nabla X_m
\\ [2mm]
&&\tau_{i,j} = \frac{2}{3} \mu \delta_{i,j} \frac{\partial {u_k}}{\partial x_k} - \mu \Big(\frac{u_i}{\partial x_j} + \frac{u_j}{\partial x_i}\Big)
\\ [2mm]
&&\boldsymbol{\mathcal{Q}} =  \sum_m h_m \boldsymbol{\mathcal{F}}_{m}  - \lambda \nabla T
\end{eqnarray*}
Details about the computation of $\mu$, $\lambda$ and $D_{m,mix}$ are discussed in Section~\ref{subs:EGLIB}.
With these specializations, the velocity constraint (Eq~\ref{eq:veloconstr}) becomes:
\begin{eqnarray}
\nabla \cdot \boldsymbol{u} \;\;  \equiv \; S&=& \frac{1}{\rho c_{p} T}(\nabla \cdot \lambda \nabla T - \sum_m  \boldsymbol{\mathcal{F}}_{m} \cdot \nabla h_m) \nonumber
\\
&-& \frac{W}{\rho} \sum_m \frac{1}{W_m} \nabla \cdot \boldsymbol{\mathcal{F}}_{m} + \frac{1}{\rho} \sum_m \Big( \frac{W}{W_m} -\frac{h_m(T)}{c_{p} T} \Big)\dot{\omega}_m
\end{eqnarray}


% We will talk about this later in the document...
% The resolution of the system of equations presented is performed in a fractional step framework which prohibits, in general, to numerically conserve both species and enthalpy while satisfying the equation of state (eos) Eq.~\ref{eq:eos}. To deal with this issue, a pressure correction term is added to the constraint S in Eq.~\ref{eq:veloconstr} to damp the system back onto the ambient eos (?? CHECK THAT FORMULA):
% \begin{eqnarray}
% \hat{S} = S + f \frac{c_{p} - R}{\Delta t c_{p} \hat{p}} (\hat{p} - P_0)
% \end{eqnarray}
% where $\hat{p}$ is computed via the eos Eq.~\ref{eq:eos}, $R = \mathcal{R}/W$ and $f$ is a damping factor ($<1$).

% \subsection{Transport coefficients and mixture rules: the Ern and Giovangigli approximations}
% \label{subs:EGLIB}
The mixture-averaged transport coefficients discussed above can be evaluated from transport properties of the pure species. We follow the treatment used in the EGLib library, based on the theory/approximations developed by Ern and Givangigli~\cite{Ern:1994,Ern:2004}.

The following choices are currently implemented in \pelelm\ 
\begin{itemize}
\item The viscosity, $\mu$, is estimated based \textcolor{red}{FIXME}
\item The conductivity, $\lambda$, is based on an empirical mixture formula:
\begin{eqnarray*}
\lambda = \Big( \sum_m X_m (\lambda_m)^{\alpha} \Big)^{1/\alpha}, \;\; \alpha = 1/4
\end{eqnarray*}
\item The flux diffusion flux is approximated using the diagonal matrix $diag(\widetilde{ \Upsilon})$, where:
\begin{eqnarray}
\widetilde{ \Upsilon}_m = \frac{W_m}{W} D_{m,mix}, \;\;\;\mbox{where} \;\;\; D_{m,mix} = \frac{1-Y_m}{ \sum_{j \neq m} X_j / \mathcal{D}_{m,j}}
\label{eq:dmix}
\end{eqnarray}
This leads to a mixture-averaged approximation that is similar to that of Hirschfelder-Curtiss~\cite{Hirschfelder:1954}:
\begin{eqnarray*}
\rho Y_m \boldsymbol{V_m} = - \rho D_{m,mix} \frac{W_m}{W} \nabla X_m 
\end{eqnarray*}
\end{itemize}
Note that with these definitions, there is no guarantee that $\sum \boldsymbol{\mathcal{F}}_{m} = 0$, as
required for mass conservation. As discussed in Section~\ref{AlgoDetails}, an arbitrary ``correction flux,'' consistent with the mixture-averaged diffusion approximation, is added in \pelelm\ to enforce conservation.


\subsection{Pure species transport properties}
The mixture-averaged transport coefficients require expressions for the pure species binary transport coefficients.  These, in turn, depend upon the forces of interaction between colliding molecules, which are complex functions of the shape and properties of each binary pair of species involved, as well as of their environment, intermolecular distance, etc. In practice, these interactions are usually described by a Lennard-Jones 6-12 potential (for non polar molecules, Stockmayer potential otherwise) that relates the evolution of the potential energy of the pair of species to their intermolecular distance. Here, the single component viscosities and binary diffusion coefficients are given by~\cite{Hirschfelder:1954}:
\begin{eqnarray}
\eta_m = \frac{5}{16} \frac{\sqrt{\pi m_m k_B T}}{\pi \sigma^2_m \Omega^{(2,2)*}},
\hspace{4mm}
\mathcal{D}_{m,j} = \frac{3}{16}\frac{\sqrt{2 \pi k^3_B T^3/m_{m,j}}}{P_0 \pi \sigma^2_{m,j} \Omega^{(1,1)*}}
\label{binary}
\end{eqnarray}
where $k_B$ is the Boltzmann constant, $\sigma_m$ is the Lennard-Jones collision diameter and $m_m (= W_k/\mathcal{A})$ is the molecular mass of species $m$. $m_{m,j}$ is the reduced molecular mass and $\sigma_{m,j}$ is the reduced collision diameter of the $(m,j)$ pair, given by:
\begin{eqnarray}
m_{m,j} = \frac{m_m m_j }{ (m_m + m_j)},
\hspace{4mm}
\sigma_{m,j} = \frac{1}{2} \zeta^{-\frac{1}{6}}(\sigma_m + \sigma_j)
\label{redCollision}
\end{eqnarray}
where $\zeta=1$ if the partners are either both polar or both nonpolar, but in the case of a polar molecule ($p$) interacting with a nonpolar ($n$) molecule:
\begin{eqnarray*}
\zeta=1 + \frac{1}{4} \alpha^*_n (\mu^*_p)^2 \sqrt{\frac{\epsilon_p}{\epsilon_n}}
\end{eqnarray*}
with $ \alpha^*_n = \alpha_n / \sigma^3_n$ the reduced polarizability of the nonpolar molecule and  $\mu^*_p = \mu_p/\sqrt{\epsilon_p \sigma^3_p}$ the reduced dipole moment of the polar molecule, expressed in function of the Lennard-Jones potential $\epsilon_p$ of the $p$ molecule.

Both quantities appearing in~\ref{binary} rely upon the evaluation of \emph{collision integrals} $\Omega^{(\cdot,\cdot)*}$, which account for inter-molecular interactions, and are usually tabulated in function of reduced variables~\cite{Monchick:1961}:
\begin{itemize}
\item $\Omega^{(2,2)*}$ is tabulated in function of a reduced temperature ($T^*_m $) and a reduced dipole moment ($\delta^*_m$), given by:
\begin{eqnarray*}
T^*_m = \frac{k_BT}{\epsilon_m},
\hspace{4mm}
\delta^*_m = \frac{1}{2} \frac{\mu^2_m}{\epsilon_m \sigma^3_m}
\end{eqnarray*}
%where $\epsilon_m$ is the Lennard-Jones potential well depth and $\mu_m$ is the dipole moment of species $m$. 
\item $\Omega^{(1,1)*}$ is tabulated in function of a reduced temperature ($T^*_{m,j} $) and a reduced dipole moment ($\delta^*_{m,j}$), given by:
\begin{eqnarray*}
T^*_{m,j} = \frac{k_BT}{\epsilon_{m,j}},
\hspace{4mm}
\delta^*_{m,j} = \frac{1}{2} \frac{\mu^2_{m,j}}{\epsilon_{m,j} \sigma^3_{m,j}}
\end{eqnarray*}
where the reduced collision diameter of the pair ($\sigma_{m,j}$) is given by \ref{redCollision}; and the Lennard-Jones potential $\epsilon_{m,j}$ and dipole moment $\mu_{m,j}$ of the $(m,j)$ pair are given by:
\begin{eqnarray*}
\frac{\epsilon_{m,j}}{k_B} = \zeta^2 \sqrt{\frac{\epsilon_m}{k_B} \frac{\epsilon_j}{k_B}},
\hspace{4mm}
\mu^2_{m,j} = \xi \mu_m \mu_j 
\end{eqnarray*}
with $\xi = 1$ if $\zeta = 1$ and $\xi = 0$ otherwise.
\end{itemize}

The expression for the pure species thermal conductivities are more complex. They are assumed to be composed of translational, rotational and vibrational contributions~\cite{Warnatz:}:
\begin{eqnarray*}
\lambda_m = \frac{\eta_m}{W_m} (f_{tr}C_{v,tr} + f_{rot}C_{v,rot} + f_{vib}C_{v,vib})
\end{eqnarray*}
where
\begin{eqnarray*}
&&f_{tr} = \frac{5}{2}\Big(1-\frac{2}{\pi} \frac{C_{v,rot}}{C_{v,tr}} \frac{A}{B} \Big)
\\
&&f_{rot} = \frac{\rho \mathcal{D}_{m,m}}{\eta_m} \Big( 1 + \frac{2}{\pi} \frac{A}{B}  \Big)
 \\
&&f_{vib} = \frac{\rho \mathcal{D}_{m,m}}{\eta_m}
\end{eqnarray*}
and
\begin{eqnarray*}
A = \frac{5}{2} - \frac{\rho \mathcal{D}_{m,m}}{\eta_m},
\hspace{4mm}
B = Z_{rot} + \frac{2}{\pi} \Big( \frac{5}{3} \frac{C_{v,rot}}{\mathcal{R}} + \frac{\rho \mathcal{D}_{m,m}}{\eta_m} \Big)
\end{eqnarray*}
The molar heat capacities $C_{v,\cdot}$ depend on the molecule shape. In the case of a linear molecule:
\begin{eqnarray*}
\frac{C_{v,tr}}{\mathcal{R}} = \frac{3}{2},
\hspace{1.5em}
\frac{C_{v,rot}}{\mathcal{R}} = 1,
\hspace{1.5em} 
{C_{v,vib}} = C_v - \frac{5}{2} \mathcal{R}
\end{eqnarray*}
In the case of a nonlinear molecule, the expressions are
\begin{eqnarray*}
\frac{C_{v,tr}}{\mathcal{R}} = \frac{3}{2},
\hspace{1.5em} 
\frac{C_{v,rot}}{\mathcal{R}} =  \frac{3}{2},
\hspace{1.5em} 
{C_{v,vib}} = C_v - 3 \mathcal{R}
\end{eqnarray*}
For single-atom molecules the thermal conductivity reduces to:
\begin{eqnarray*}
\lambda_m = \frac{\eta_m}{W_m} (f_{tr}C_{v,tr} ) = \frac{15 \, \eta_m \mathcal{R}}{4 \, W_m}
\end{eqnarray*}
Finally, $Z_{rot}$ is the rotational relaxation number, a parameter given by~\cite{Parker:}:
\begin{eqnarray*}
Z_{rot}(T) = Z_{rot} (298) \frac{F(298)}{F(T)}
\end{eqnarray*}
with 
\begin{eqnarray*}
F(T) = 1 + \frac{\pi^{(3/2)}}{2} \sqrt{\frac{\epsilon/k_B}{T} } + \Big( \frac{\pi^2}{4} +2 \Big) \Big( \frac{\epsilon/k_B}{T} \Big) + \pi^{(3/2)}\Big( \frac{\epsilon/k_B}{T} \Big)^{(3/2)} 
\end{eqnarray*}

In \pelelm\ the pure species transport properties are evaluated with EGLib functions.  EGLib requires polynomial fits of the logarithm of each quantity versus the logarithm of the temperature.
\begin{eqnarray*}
ln(q_m) = \sum_{n=1}^4 a_{q,m,n} \, ln(T)^{(n-1)} 
\end{eqnarray*}
where $q_m$ represents $\eta_m$, $\lambda_m$ or $D_{m,j}$. These fits are generated as part of a preprocessing step managed by the tool \fuego\ based on the formula (and input data) discussed above.  In particular, for each chemical species represented, $j$, the preprocessor tool requires the following data: XXXXXX.  The role of \fuego\ to preprocess the model parameters for transport as well as chemical kinetics and thermodynamics, is discussed in some detail in Chapter~\ref{ch:fuego}.

%%%
\section{Chemical kinetics and the reaction source term}
\label{ChemKinetics}
Chemistry in combustion systems involves a number of species interacting through multiple types of reactions. A wide variety of chemical kinetics models can be represented as a set of $M_r$ elementary reaction steps that involve $N_s$ species.
\begin{eqnarray*}
\sum_{m=1}^{N_s} \nu_{m,j}'[X_m] \rightleftharpoons \sum_{m=1}^{N_s} \nu_{m,j}''[X_m],\quad for \quad j \in [1,M_r] 
\label{IntroKM1}
\end{eqnarray*}
where $[X_m]$ stands for the species $m$ molar concentration and $\nu_{m,j}'$, $\nu_{m,j}''$ are the molar stoichiometric coefficients of species $m$ in each side of reaction $j$. For such a system, all $M_r$ reaction rates have to be considered in describing the temporal evolution of the concentration of the species involved. The rate of a reaction $j$ ($R_j$) can be expressed in terms of the rate laws of the forward ($k_{f,j}$) and backward ($k_{r,j}$) reactions, as:
\begin{eqnarray*} 
R_{j} = k_{f,j}\prod_{m=1}^{N_s}  [X_{m}]^{\nu_{m,j}'}-k_{r,j}\prod_{m=1}^{N_s} [X_{m}]^{\nu_{m,j}''}
\end{eqnarray*}
The net molar production rate, $ \dot{\omega}_m$, in Eq.~\ref{eq:species} of species $m$ is obtained by
collating the rate of creation and destruction over reactions:
\begin{eqnarray*}
\dot{\omega}_m = \sum_{j=1}^{M_r} \nu_{m,j} R_j 
\label{IntroKM3}
\end{eqnarray*}
where $\nu_{m,j} =\nu_{m,j}'' - \nu_{m,j}'$. Expressions for the reaction rates coefficients $k_j$ depend on the type of reaction considered. \pelelm \; relies on the CHEMKIN Arrhenius reaction format:
\begin{eqnarray*}
k_f = AT^{\beta} exp \left( \frac{-E_a}{RT}\right)
\end{eqnarray*}
where $A$ is the pre-exponential (frequency) factor, $\beta$ is the temperature exponent and $E_a$ is the activation energy. The CHEMKIN format allows for a number of specializations of this format to represent pressure dependencies and third-body enhancements -- See the CHEMKIN Manual or Cantera website for additional information~\cite{Kee:1989,cantera}.

Most fundamental Arrhenius reactions are bidirectional, and typically only the forward rates are specified. In this case, the balance of forward and reverse rates are dictacted by equilibrium thermodynamics, via the equilibrium ``constant'', $K_{c}$ that in a low Mach system, is only a function of temperature:
\begin{eqnarray*}
&&k_{r,j} = \frac{k_{f,j}}{K_{c}(T)} \;\;\; \mbox{where} \;\;\; K_c=K_{p_i} \left( \frac{P_{0}}{RT} \right)^{\sum_{k=1}^{N_s} \nu_{ki}}
\\
&&\mbox{and} \;\;\; K_{p_i}=\exp \left( \frac{\Delta {S_j}^{0}}{R} - \frac{\Delta {H_j}^{0}}{RT} \right)
\end{eqnarray*}
$\Delta H_j$ and $\Delta S_j$ are the change in enthaply and entropy of the reaction $j$, and $P_0$ is the ambient thermodynamic pressure.

The preprocessing tool \fuego\ is used to import the parameters of the chemistry model for use by \pelelm.  The model parameters are loaded into a Python database, and are used to automatically generate highly efficient C source code that is linked into a \pelelm\ executable for evaluating all the necessary rate expressions.  Each model is fully contained in a single source file and can be linked separately to post-processing analysis tools as well.

%%%
\section{Thermodynamic properties}
\label{ThermoProp}
TODO


%%%
\section{The \pelelm  \; temporal integration}
\subsection{Overview}
The basic discretization combines a simplified spectral deferred correction (SDC) coupling of chemistry and transport \cite{LMC_SDC} with a density-weighted approximate projection method for low Mach number flow \cite{DayBell:2000}.  The projection method implements a constrained evolution on the velocity field via the SDC iterations, which ensures that the update simultaneously satisfies the  equation of state and discrete conservation of mass and total enthalpy.
A time-explicit approach is used for advection; faster diffusion and chemistry processes are treated time-implicitly, and iteratively coupled together within the deferred corrections strategy. Since the low Mach system does not support acoustic waves, the time step size is governed by a CFL constraint based on advective transport.
The integration algorithm is second-order accurate in space and time.


\subsection{Algorithm details}
\label{AlgoDetails}
TODO: get stuff from 1D code with electric field ?

%The performance of the numerical scheme for direct numerical simulation of premixed flame systems in regimes comparable to the present study was examined in \cite{Aspden08b}.  An {\em effective} Kolmogorov length scale, $\eta_{\mbox{\it eff}}$, was formulated, which measures the actual Kolmogorov length scale realised in a simulation at a given resolution. Here, the most computationally demanding simulation, having the highest turbulence levels, has a computational cell width that is approximately 1.27 times the Kolmogorov length scale, $\eta$.  In this case, the numerical scheme produces $\eta_{\mbox{\it eff}}/\eta\,$$<\,$1.03.  All other cases were better resolved.
