%%%
\section{The low Mach number flow equations}
\newcommand{\etal}{{\it et al.}}

\pelelm\ solves the reacting Navier-Stokes flow equations in the \emph{low Mach number} regime~\cite{DayBell:2000,rehm1978equations,Majda:1985}. In the low Mach number regime, the characteristic fluid velocity is small compared to the sound speed, and the effect of acoustic wave propagation is unimportant to the overall dynamics of the system. As such, acoustic wave propagation is mathematically removed from the equations of motion, allowing for a time step based on an advective CFL condition and leading to an increase in the allowable time step of order $1/M$ over an explicit, fully compressible method ($M$ is the Mach number). In this framework, the total pressure is decomposed into the sum of a constant (ambient) thermodynamic pressure $P_0$ and a perturbational pressure $\pi$ such that $\pi/P_0 = \mathcal{O} (M^2)$. 

The considered set of equations is a system of PDEs with Advection-Diffusion-Reaction (ADR) processes constrained by an equation of state in the form of a divergence constraint on the velocity. The evolution equations for the velocity and thermodynamic variables reads:
\begin{eqnarray}
\frac{\partial (\rho \boldsymbol{u})}{\partial t} + 
\nabla \cdot \left(\rho  \boldsymbol{u} \boldsymbol{u} + \tau \right)
&=& -\nabla \pi + \rho \, \boldsymbol{F}  ,
\label{eq:mom}
\\
\frac{\partial (\rho Y_m)}{\partial t} +
\nabla \cdot \left( \rho Y_m \boldsymbol{u} + \boldsymbol{\mathcal{F}}_{m} \right)
&=& \rho \, \dot{\omega}_m,
\label{eq:species}
\\
\frac{ \partial (\rho h)}{ \partial t} +
\nabla \cdot \left( \rho h \boldsymbol{u} + \boldsymbol{\mathcal{Q}} \right) &=& 0 ,
\label{eq:enthalpy}
\end{eqnarray}
where $\rho$ is the density, $\boldsymbol{u}$ is the velocity, $h$ is the mass-weighted enthalpy, $T$ is temperature and $Y_m$ is the mass fraction of species $m$. $\dot{\omega}_m$ is the molar production rate for species $m$, the modeling of which will be described in Section~\ref{ChemKinetics}. $\tau$ is the stress tensor, $\boldsymbol{\mathcal{Q}}$ is the heat flux vector and $\boldsymbol{\mathcal{F}}_m$
%$\,=\,$$- \rho \boldsymbol{\mathcal{D}}_i \nabla X_{i}$ 
are the species diffusion vectors. These transport fluxes require the evaluation of transport coefficients (e.g., the viscosity $\mu$, the conductivity $\lambda$ and the diffusivity matrix $D$) which are computed using the library EGLIB~\cite{EGLIB}, as will be described in more depth in Section~\ref{DifFluxesEGLIB}. The momentum source term, $\boldsymbol{F}$, is a long-wavelength forcing term designed to establish and maintain turbulence with the desired properties. Note that the definition of enthalpy includes the standard enthalpy of formation, so there is no net change to $h$ due to reactions.

These evolution equations are supplemented by an equation of state:
\begin{eqnarray}
P_0=\rho \mathcal{R} T \sum_m \frac{Y_m}{W_m}
\label{eq:eos}
\end{eqnarray}
where $W_m$ is the species $m$ molecular weight; and by a relationship between enthalpy, species and temperature:
\begin{eqnarray}
h=\sum_m Y_m h_m(T)
\end{eqnarray}
where $h_m$ is the species $m$ enthalpy. 

Neither species diffusion nor reactions redistribute the total mass, hence we have $\sum_m \boldsymbol{\mathcal{F}}_m = 0$ and $\sum_m \dot{\omega}_m = 0$. Thus, summing the species equations and using $\sum_m Y_m = 1$ we obtain the continuity equation from Eq.~\ref{eq:species}:
\begin{eqnarray}
\frac{\partial \rho}{\partial t} + \nabla \cdot \rho \boldsymbol{u} = 0
\end{eqnarray}

The low Mach number constraint is derived by differentiating the equation of state along particle paths (see~\cite{pember-flame}):
\begin{eqnarray}
\nabla \cdot \boldsymbol{u} = \frac{1}{T}\frac{DT}{Dt} + W \sum_m \frac{1}{W_m} \frac{DY_m}{Dt} = S
\label{eq:veloconstr}
\end{eqnarray}
where $W$ is the mean mixture molecular weight. An expression for S in the simplified framework described hereafter will be given in Section~\ref{SumUpEq}





%%%
\section{The diffusive fluxes}
\label{DifFluxesEGLIB}
\subsection{Expressions for the diffusive fluxes}
\label{sub:DifFluxes}
Expressions for the transport fluxes appearing in Eqs.~(\ref{eq:mom}-\ref{eq:enthalpy}) can be approximated in the Enskog-Chapman expansion as~\cite{Ern:1994multicomponent}:
 \begin{eqnarray}
\tau_{i,j} = - \Big(\kappa - \frac{2}{3} \mu \Big) \delta_{i,j} \frac{\partial {u_k}}{\partial x_k} - \mu \Big(\frac{u_i}{\partial x_j} + \frac{u_j}{\partial x_i}\Big)
\\
\boldsymbol{\mathcal{F}}_{m} = \rho Y_m \boldsymbol{V_m}
\\
\boldsymbol{\mathcal{Q}} =  \sum_m h_m \boldsymbol{\mathcal{F}}_{m}  - \lambda' \nabla T - P_0 \sum_m \theta_m \boldsymbol{d_m}
\end{eqnarray}
where the $m$ species diffusion velocities vectors $\boldsymbol{V_m}$ are given by:
 \begin{eqnarray}
\boldsymbol{V_m} = - \sum_j  {D}_{m,j} \boldsymbol{d_j} - \theta_m \nabla ln(T)
\end{eqnarray}
The vectors $\boldsymbol{d_m}$ incorporate the effects of various state variable gradients and external forces and are given by~\cite{Ern:1994multicomponent}:
 \begin{eqnarray}
\boldsymbol{d_m} = \nabla X_m + (X_m -Y_m) \frac{\nabla P_0}{P_0}
\label{dmeqs}
\end{eqnarray}
The $\theta_m$ are thermal diffusion vectors, associated with the Soret (mass concentration flux due to an energy gradient) and Dufour (the energy flux due to a mass concentration gradient) effects. Alternatively, the thermal diffusion \emph{ratios} $\chi_m$ may be preferred~\cite{Ern:1994multicomponent} and the diffusion velocities and energy flux recast as:
 \begin{eqnarray}
\boldsymbol{V_m} = - \sum_j  {D}_{m,j} ( \boldsymbol{d_j} + \chi_j \nabla ln(T))
\\
\boldsymbol{\mathcal{Q}} =  \sum_m h_m \boldsymbol{\mathcal{F}}_{m}  - \lambda \nabla T + P_0 \sum_m \chi_m \boldsymbol{V_m}
\end{eqnarray}
where  $\theta_m = \sum_j  {D}_{m,j} \chi_j$ and $\lambda' \nabla T = \lambda \nabla T + P_0 \sum_m \theta_m \nabla ln(T)$.

As can be seen, the expression for these fluxes relies upon several transport coefficients that need to be evaluated: the shear viscosity $\mu$, the volume viscosity $\kappa$, the diffusion matrix $ {D}_{m,j} $ and either the thermal diffusion vector $\boldsymbol{\theta}$ and the partial thermal conductivity $\lambda'$ or the thermal diffusion ratios $\boldsymbol{\chi}$ and the thermal conductivity $\lambda$. However, in the present framework several effects are neglected, thus simplifying the fluxes evaluation:
\begin{enumerate}
\item In a low Mach limit, the volume viscosity $\kappa$ is thought to have a very limited impact
\item Likewise, pressure effects in Eqs.~\ref{dmeqs} are neglected since pressure variations are supposed to be very small
\item Finally, we neglect the Dufour and Soret effects so that $\theta_m = 0 = \chi_m$ and $\lambda = \lambda '$
\end{enumerate}

\subsection{Pure species transport properties}
With the usual Mason and Monchick approximations~\cite{Mason:1962}, the evaluation of the aforementioned transport coefficients reduces to the computation of pure species transport properties. These, in turn, depend upon the forces of interaction between colliding molecules. In reality, molecular interactions are complex functions of the shape and properties of the pair of species involved, as well as of their environment, intermolecular distance, etc. In practice, they are usually described by a Lennard-Jones 6-12 potential (for non polar molecules, Stockmayer potential otherwise) that relates the evolution of the potential energy of the pair of species to their intermolecular distance.

In this standard framework, the single component viscosities and binary diffusion coefficients are given by~\cite{Hirschfelder:1954}:
\begin{eqnarray}
\eta_m = \frac{5}{16} \frac{\sqrt{\pi m_m k_B T}}{\pi \sigma^2_m \Omega^{(2,2)*}}
\label{muCoefs}
\\
 \mathcal{D}_{m,j} = \frac{3}{16}\frac{\sqrt{2 \pi k^3_B T^3/m_{m,j}}}{P_0 \pi \sigma^2_{m,j} \Omega^{(1,1)*}}
 \label{difCoefs}
\end{eqnarray}
where $k_B$ is the Boltzmann constant, $\sigma_m$ is the Lennard-Jones collision diameter and $m_m (= W_k/\mathcal{A})$ is the molecular mass of species $m$. $m_{m,j}$ is the reduced molecular mass and $\sigma_{m,j}$ is the reduced collision diameter of the $(m,j)$ pair, given by:
\begin{eqnarray}
m_{m,j} = \frac{m_m m_j }{ (m_m + m_j)}
\\
\sigma_{m,j} = \frac{1}{2} \zeta^{-\frac{1}{6}}(\sigma_m + \sigma_j)
\label{redCollision}
\end{eqnarray}
where $\zeta=1$ if the partners are either both polar or both nonpolar, but in the case of a polar molecule ($p$) interacting with a nonpolar ($n$) molecule:
\begin{eqnarray}
\zeta=1 + \frac{1}{4} \alpha^*_n (\mu^*_p)^2 \sqrt{\frac{\epsilon_p}{\epsilon_n}}
\end{eqnarray}
with $ \alpha^*_n = \alpha_n / \sigma^3_n$ the reduced polarizability of the nonpolar molecule and  $\mu^*_p = \mu_p/\sqrt{\epsilon_p \sigma^3_p}$ the reduced dipole moment of the polar molecule, expressed in function of the Lennard-Jones potential $\epsilon_p$ of the $p$ molecule.

Both quantities expressed by~\ref{muCoefs} and~\ref{difCoefs} rely upon the evaluation of \emph{collision integrals} $\Omega^{(\cdot,\cdot)*}$ -accounting for inter-molecular interactions, which are usually tabulated in function of reduced variables~\cite{Monchick:1961}:
\begin{itemize}
\item $\Omega^{(2,2)*}$ is tabulated in function of a reduced temperature ($T^*_m $) and a reduced dipole moment ($\delta^*_m$), given by:
\begin{eqnarray}
T^*_m = \frac{k_BT}{\epsilon_m}
\\
\delta^*_m = \frac{1}{2} \frac{\mu^2_m}{\epsilon_m \sigma^3_m}
\end{eqnarray}
%where $\epsilon_m$ is the Lennard-Jones potential well depth and $\mu_m$ is the dipole moment of species $m$. 
\item $\Omega^{(1,1)*}$ is tabulated in function of a reduced temperature ($T^*_{m,j} $) and a reduced dipole moment ($\delta^*_{m,j}$), given by:
\begin{eqnarray}
T^*_{m,j} = \frac{k_BT}{\epsilon_{m,j}}
\\
\delta^*_{m,j} = \frac{1}{2} \frac{\mu^2_{m,j}}{\epsilon_{m,j} \sigma^3_{m,j}}
\end{eqnarray}
where the reduced collision diameter of the pair ($\sigma_{m,j}$) is given by \ref{redCollision}; and the Lennard-Jones potential $\epsilon_{m,j}$ and dipole moment $\mu_{m,j}$ of the $(m,j)$ pair are given by:
\begin{eqnarray}
\frac{\epsilon_{m,j}}{k_B} = \zeta^2 \sqrt{\frac{\epsilon_m}{k_B} \frac{\epsilon_j}{k_B}}
\\
\mu^2_{m,j} = \xi \mu_m \mu_j 
\end{eqnarray}
with $\xi = 1$ if $\zeta = 1$ and $\xi = 0$ otherwise.
\end{itemize}

The expression for the pure species thermal conductivities are more complex. They are assumed to be composed of translational, rotational and vibrational contributions~\cite{Warnatz:}:
\begin{eqnarray}
\lambda_m = \frac{\eta_m}{W_m} (f_{tr}C_{v,tr} + f_{rot}C_{v,rot} + f_{vib}C_{v,vib})
\end{eqnarray}
where
\begin{eqnarray}
f_{tr} = \frac{5}{2}\Big(1-\frac{2}{\pi} \frac{C_{v,rot}}{C_{v,tr}} \frac{A}{B} \Big)
\\
 f_{rot} = \frac{\rho \mathcal{D}_{m,m}}{\eta_m} \Big( 1 + \frac{2}{\pi} \frac{A}{B}  \Big)
 \\
 f_{vib} = \frac{\rho \mathcal{D}_{m,m}}{\eta_m}
\end{eqnarray}
and
\begin{eqnarray}
A = \frac{5}{2} - \frac{\rho \mathcal{D}_{m,m}}{\eta_m}
\\
B = Z_{rot} + \frac{2}{\pi} \Big( \frac{5}{3} \frac{C_{v,rot}}{\mathcal{R}} + \frac{\rho \mathcal{D}_{m,m}}{\eta_m} \Big)
\end{eqnarray}
The molar heat capacities $C_{v,\cdot}$ depend on the molecule shape. In the case of a linear molecule it reads:
\begin{eqnarray}
\frac{C_{v,tr}}{\mathcal{R}} = \frac{3}{2}
\\
\frac{C_{v,rot}}{\mathcal{R}} = 1
\\
{C_{v,vib}} = C_v - \frac{5}{2} \mathcal{R}
\end{eqnarray}
when in the case of a nonlinear molecule it reads:
\begin{eqnarray}
\frac{C_{v,tr}}{\mathcal{R}} = \frac{3}{2}
\\
\frac{C_{v,rot}}{\mathcal{R}} =  \frac{3}{2}
\\
{C_{v,vib}} = C_v - 3 \mathcal{R}
\end{eqnarray}
Note that if the molecule is in fact an atom, then the thermal conductivity should reduce to:
\begin{eqnarray}
\lambda_m = \frac{\eta_m}{W_m} (f_{tr}C_{v,tr} ) = \frac{\eta_m}{W_m} \Big(\frac{5}{2}  \frac{3}{2} \mathcal{R}\Big) 
\end{eqnarray}
Finally, $Z_{rot}$ is the rotational relaxation number, a parameter given by~\cite{Parker:}:
\begin{eqnarray}
Z_{rot}(T) = Z_{rot} (298) \frac{F(298)}{F(T)}
\end{eqnarray}
with 
\begin{eqnarray}
F(T) = 1 + \frac{\pi^{(3/2)}}{2} \sqrt{\frac{\epsilon/k_B}{T} } + \Big( \frac{\pi^2}{4} +2 \Big) \Big( \frac{\epsilon/k_B}{T} \Big) + \pi^{(3/2)}\Big( \frac{\epsilon/k_B}{T} \Big)^{(3/2)} 
\end{eqnarray}

In \pelelm, \; the pure species transport properties are evaluated with the preprocessing tool \emph{Fuego}, and a polynomial fit of the logarithm of the property versus the logarithm of the temperature is used to store the data in an external cpp subroutine. During a \pelelm \; run, these quantities are accessed via:
\begin{eqnarray}
ln(q_m) = \sum_{\substack{1<n<4}} a_{q,m,n} ln(T)^{(n-1)} 
\end{eqnarray}
where $q$ represents either $\eta_m$, $\lambda_m$ or $D_{m,j}$.


\subsection{Transport coefficients and mixture rules: the Ern and Giovangigli approximations}
\label{subs:EGLIB}
The mixture transport coefficients discussed in Section~\ref{sub:DifFluxes} can now be evaluated from the pure species transport properties. If any accurate formulation requires solving a series of linear systems, however, several empirical mixing rules also exist, and a good compromise between accuracy and efficiency is most often case and code dependent. In \pelelm, the EGLib library is employed, building on the multicomponent transport theory developed by Ern and Givangigli~\cite{Ern:1994}. This library implements various algorithms to evaluate mixture transport coefficients, enabling to choose between a direct inversion, efficient and rigorously derived iterative methods providing various approximations or accurate mixture-averaged empirical formulations~\cite{Ern:2004}. If it is possible to navigate between the various evaluations, the following choices have been made in \pelelm \;(for more details about the formulations, see~\cite{Ern:1994,Ern:2004}):
\begin{itemize}
\item The viscosity $\mu$ is estimated based on the first step of a conjugate-gradient method 
\item The condctivity $\lambda$ is based on an empirical formulation which reads:
\begin{eqnarray}
\lambda = \frac{1}{2} (\Xi(-1) + \Xi(1))
\end{eqnarray}
with
\begin{eqnarray}
\Xi(\alpha) = \Big( \sum_m X_m (\lambda_m)^{\alpha} \Big)^{1/\alpha}
\end{eqnarray}
\item The flux diffusion matrix ${\widetilde{D}_{m,j}} = Y_m D_{m,j}$ is approximated by the diagonal matrix $diag(\widetilde{ \Upsilon})$, where:
\begin{eqnarray}
\widetilde{ \Upsilon}_m = \frac{W_m}{W} D_{m,mix}
\\
 D_{m,mix} = \frac{1-Y_m}{ \sum_{j \neq m} X_j / \mathcal{D}_{m,j}}
\end{eqnarray}
leading to a mixture-averaged approximation of the species diffusion fluxes similar to that of Hirschfelder-Curtiss~\cite{Hirschfelder:1954}:
\begin{eqnarray}
\rho Y_m \boldsymbol{V_m} = - \rho D_{m,mix} \frac{W_m}{W} \nabla X_m 
\end{eqnarray}
Note that with this approximation, the global mass is no longer conserved. To enforce it, a "conservation diffusion velocity" is introduced in the resolution algorithm, as will be further discussed in Section~\ref{AlgoDetails}.
\end{itemize}


%%%
\section{Chemical kinetics and the reaction source term}
\label{ChemKinetics}
Chemistry in combustion systems always involves a large number of species interacting through multiple types of reactions, even for the most elementary mixtures. In practice this is modeled through a reaction scheme involving a finite number of elementary steps ($M_r$) between a finite number of necessary species ($N_s$):
\begin{eqnarray}
\sum_{\substack{1<m<N_s}} \nu_{m,j}'[X_m] \rightleftharpoons \sum_{\substack{1<m<N_s}} \nu_{m,j}''[X_m],\quad for \quad j \in [1,M_r] 
\label{IntroKM1}
\end{eqnarray}
where $[X_m]$ stands for the species $m$ molar concentration and $\nu_{m,j}'$, $\nu_{m,j}''$ are the molar stoichiometric coefficients of species $m$ in each side of reaction $j$. For such a system, all $M_r$ reaction rates have to be considered in describing the temporal evolution of the concentration of the species involved. The rate of a reaction $j$ ($R_j$) can be measured in terms of the rate laws of the forward ($k_{f,j}$) and backward ($k_{r,j}$) reactions, as:
\begin{eqnarray} 
R_{j} = k_{f,j}\prod_{\substack{1<m<N_s}}  [X_{m}]^{\nu_{m,j}'}-k_{r,j}\prod_{\substack{1<m<N_s}} [X_{m}]^{\nu_{m,j}''}
\end{eqnarray} 
Ultimately these are used to express $ \dot{\omega}_m$ for each species $m$ in Eq.~\ref{eq:species}, as:
\begin{eqnarray}
\dot{\omega}_m = \sum_{\substack{0<j<M_r}} \nu_{m,j} R_j 
\label{IntroKM3}
\end{eqnarray}
where $\nu_{m,j} =\nu_{m,j}'' - \nu_{m,j}'$. Expressions for the reaction rates coefficients $k_j$ depend on the type of reaction considered. \pelelm \; relies on the CHEMKIN reaction scheme format that supports a variety of reaction types; however, always expressed in the classical Arrhenius format:
\begin{eqnarray}
k_f = AT^{\beta} exp \left( \frac{-Ea}{RT}\right)
\end{eqnarray}
where $A$ is the pre-exponential factor or frequency factor, $\beta$ is the temperature exponent and $Ea$ is the activation energy. With this format, reactions can bear, e.g., pressure dependencies. See the CHEMKIN Manual or Cantera website for additional information~\cite{Kee:1989,cantera}.

Note that reaction information is usually provided as "one-way". Forward and backward reaction rates can be related through the {pressure equilibrium constant} $K_{p}$, a function temperature dependent, as : 
\begin{equation}
k_{r,j} = \frac{k_{f,j}}{K_{p}(T)} =\frac{k_{f}}{ {\left( \frac{P_0}{RT} \right)}^{\sum_{\substack{1<m<N_s}} \nu_{m,j}} exp \left( \frac{\Delta {S_j}^{0}}{R} - \frac{\Delta {H_j}^{0}}{RT} \right) }
\end{equation}
where $ \Delta$ stands for an enthaply ($H$) or entropy ($S$) variation between the products and the reactants of the reaction $j$ and $P_0$ is the standard atmospheric pressure (1 bar).

The preprocessing tool Fuego is used to manage the CHEMKIN input file, and expressions for an "on-the-fly" evaluation of reaction rates and species production rates during a \pelelm \; run are stored in a cpp subroutine.

%%%
\section{Thermodynamic properties}
\label{ThermoProp}
TODO


%%%
\section{Final \pelelm  \;  set of equations}
\label{SumUpEq}

\begin{eqnarray}
\frac{\partial (\rho \boldsymbol{u})}{\partial t} + 
\nabla \cdot \left(\rho  \boldsymbol{u} \boldsymbol{u} + \tau \right)
&=& -\nabla \pi + \rho \, \boldsymbol{F}  ,
\label{eq:mom}
\\
\frac{\partial (\rho Y_i)}{\partial t} +
\nabla \cdot \left( \rho Y_i \boldsymbol{u} + \boldsymbol{\mathcal{F}}_{i} \right)
&=& \rho \, \dot{\omega}_i,
\label{eq:species}
\\
\frac{ \partial (\rho h)}{ \partial t} +
\nabla \cdot \left( \rho h \boldsymbol{u} + \boldsymbol{\mathcal{Q}} \right) &=& 0 ,
\label{eq:enthalpy}
\end{eqnarray}
with ( !!! I dont understand why grad Ym in 2nd eq ??)
 \begin{eqnarray}
\tau_{i,j} = \frac{2}{3} \mu \delta_{i,j} \frac{\partial {u_k}}{\partial x_k} - \mu \Big(\frac{u_i}{\partial x_j} + \frac{u_j}{\partial x_i}\Big)
\\
\boldsymbol{\mathcal{F}}_{m} = \rho Y_m \boldsymbol{V_m} = - \rho D_{m,mix} \nabla Y_m
\\
\boldsymbol{\mathcal{Q}} =  \sum_m h_m \boldsymbol{\mathcal{F}}_{m}  - \lambda \nabla T
\end{eqnarray}
with $\mu$, $\lambda$ and $D_{mix}$ computed as in Section~\ref{subs:EGLIB}. In this context, the velocity constraint reads:
\begin{eqnarray}
\nabla \cdot \boldsymbol{u} = S= \frac{1}{\rho c_p T}(\nabla \cdot \lambda \nabla T - \sum_m  \boldsymbol{\mathcal{F}}_{m} \cdot \nabla h_m)
\\
- \frac{W}{\rho} \sum_m \frac{1}{W_m} \nabla \cdot \boldsymbol{\mathcal{F}}_{m} + \frac{1}{\rho} \sum_m \Big( \frac{W}{W_m} -\frac{h_m(T)}{c_p T} \Big)\dot{\omega}_m
\end{eqnarray}



The resolution of the system of equations presented is performed in a fractional step framework which prohibits, in general, to numerically conserve both species and enthalpy while satisfying the equation of state (eos) Eq.~\ref{eq:eos}. To deal with this issue, a pressure correction term is added to the constraint S in Eq.~\ref{eq:veloconstr} to damp the system back onto the ambient eos (?? CHECK THAT FORMULA):
\begin{eqnarray}
\hat{S} = S + f \frac{c_p - R}{\Delta t c_p \hat{p}} (\hat{p} - P_0)
\end{eqnarray}
where $\hat{p}$ is computed via the eos Eq.~\ref{eq:eos}, $R = \mathcal{R}/W$ and $f$ is a damping factor ($<1$).

%%%
\section{The \pelelm  \; temporal integration}
\subsection{Overview}
The basic discretization combines a simplified spectral deferred correction (SDC) coupling of chemistry and transport \cite{LMC_SDC} with a density-weighted approximate projection method for low Mach number flow \cite{DayBell:2000}.  The projection method implements a constrained evolution on the velocity field via the SDC iterations, which ensures that the update simultaneously satisfies the  equation of state and discrete conservation of mass and total enthalpy.
A time-explicit approach is used for advection; faster diffusion and chemistry processes are treated time-implicitly, and iteratively coupled together within the deferred corrections strategy. Since the low Mach system does not support acoustic waves, the time step size is governed by a CFL constraint based on advective transport.
The integration algorithm is second-order accurate in space and time.


\subsection{Algorithm details}
\label{AlgoDetails}
TODO: get stuff from 1D code with electric field ?

%The performance of the numerical scheme for direct numerical simulation of premixed flame systems in regimes comparable to the present study was examined in \cite{Aspden08b}.  An {\em effective} Kolmogorov length scale, $\eta_{\mbox{\it eff}}$, was formulated, which measures the actual Kolmogorov length scale realised in a simulation at a given resolution. Here, the most computationally demanding simulation, having the highest turbulence levels, has a computational cell width that is approximately 1.27 times the Kolmogorov length scale, $\eta$.  In this case, the numerical scheme produces $\eta_{\mbox{\it eff}}/\eta\,$$<\,$1.03.  All other cases were better resolved.
