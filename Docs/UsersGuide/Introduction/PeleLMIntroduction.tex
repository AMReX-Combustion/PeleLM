\section{History of \pelelm}

\pelelm\ describes the evolution of chemically reacting low Mach number flows.
The core libraries for managing the subcycling adaptive mesh refinement (AMR)
grids and communication are found in the \amrex\ library
(see \url{https://github.com/AMReX-Codes/amrex}).

\pelelm\ also uses source code and base classes for integrating the
variable-density incompressible Navier-Stokes equations found in \iamr\ 
(see \url{https://github.com/AMReX-Codes/IAMR}).

The core algorithms in \pelelm\ are described in a series of papers:
\begin{itemize}

\item {\it A conservative, thermodynamically consistent numerical approach for low Mach number 
combustion. I. Single-level integration},
A.~Nonaka, J.~B.~Bell, and M.~S.~Day.  Submitted for publication, 2017.
\url{https://ccse.lbl.gov/Publications/nonaka/LMC_Pressure.pdf}

\item {\it A Deferred Correction Coupling Strategy for Low Mach Number Flow with Complex Chemistry},
A.~Nonaka, J.~B.~Bell, M.~S.~Day, C.~Gilet, A.~S.~Almgren, and M.~L.~Minion,
Combust. Theory and Modelling, 16(6), 1053-1088, 2012. 
\url{http://www.tandfonline.com/doi/abs/10.1080/13647830.2012.701019} \cite{LMC_SDC}

\item {\it Numerical Simulation of Laminar Reacting Flows with Complex Chemistry},
M.~S.~Day and J.~B.~Bell,
Combust. Theory Modelling 4(4) pp.535-556, 2000.
\url{http://www.tandfonline.com/doi/abs/10.1088/1364-7830/4/4/309} \cite{DayBell:2000}

\item {\it An Adaptive Projection Method for Unsteady, Low-Mach Number Combustion}, 
R.~B.~Pember, L.~H.~Howell, J.~B.~Bell, P.~Colella, W.~Y.~Crutchfield, W.~A.~Fiveland, and J.~P.~Jessee,
Comb. Sci. Tech., 140, pp. 123-168, 1998.
\url{http://www.tandfonline.com/doi/abs/10.1080/00102209808915770} \cite{pember-flame}

\item {\it A Conservative Adaptive Projection Method for the Variable Density Incompressible Navier-Stokes Equations},
A.~S.~Almgren, J.~B.~Bell, P.~Colella, L.~H.~Howell, and M.~L.~Welcome,
J.~Comp.~Phys., 142, pp. 1-46, 1998.
\url{http://www.sciencedirect.com/science/article/pii/S0021999198958909} \cite{IAMR}

\end{itemize}         

\section{Brief Overview of Low Speed Approximations}

There are many low speed formulations of the equations of hydrodynamics
in use, each with their own applications.  All of these methods share in
common a constraint equation on the velocity field that augments the
equations of motion.  

\subsection{Incompressible Hydrodynamics}

The simplest low Mach number approximation is incompressible
hydrodynamics. This approximation is formally the $M \rightarrow 0$
limit of the Navier-Stokes equations. In incompressible hydrodynamics,
the velocity satisfies a constraint equation:
\begin{equation}
\nabla \cdot \Ub = 0
\end{equation}
which acts to instantaneously equilibrate the flow, thereby filtering
out soundwaves.  The constraint equation implies that
\begin{equation}
D\rho/Dt = 0
\end{equation}
(through the continuity equation) which says that the density is
constant along particle paths. This means that there are no
compressibility effects modeled in this approximation.

\subsection{Low-Mach Number Combustion}

In the low Mach number combustion model, the pressure is decomposed
into a dynamic, $\pi$, and thermodynamic component, $p_0$, the ratio
of which is $O(M^2)$. The total pressure is replaced everywhere by the
thermodynamic pressure, except in the momentum equation. This
decouples the pressure and density and filters out the sound
waves. Large amplitude density and temperature fluctuations are
allowed. The only requirement is that the total pressure stay close to
the background pressure, which is assumed constant. This requirement
can be expressed as:
\begin{equation}
p = p_0
\end{equation}
and differentiating this along particle paths leads to a constraint on
the velocity field: 
\begin{equation}
\nabla \cdot \Ub = S 
\end{equation}
This looks like the constraint for incompressible hydrodynamics, but
now we have a source term, $S$, representing the local compressibility
effects due to the energy generation and thermal diffusion.  Since the
background pressure is taken to be constant, we cannot model flows
that cover a large fraction of a pressure scale height. However, this
method is ideal for exploring the physics of flames.

\subsection{Other Models}
We defer discussion of anelastic and pseudo-incompressible methods for now,
as they are more relevant large-gravity or large-scale flows with 
density and pressure stratification.  For more detailed discussion,
refer to our \maestro\ code
(\url{http://amrex-astro.github.io/MAESTRO/index.html}).

%% \subsection{Anelastic Hydrodynamics}

%% In the anelastic approximation small amplitude thermodynamic
%% perturbations are carried with respect to a static hydrostatic
%% background (described by density $\rho_0$).  The density perturbation
%% is ignored in the continuity equation, resulting in a constraint
%% equation:
%% \begin{equation}
%% \nabla \cdot (\rho_0 \Ub) = 0
%% \end{equation}
%% This properly captures the compressibility effects due to the
%% stratification of the background. Because there is no evolution
%% equation for the perturbational density, approximations are made to
%% the buoyancy term in the momentum equation.

%% \subsection{Pseudo-Incompressible Methods}

%% The pseudo-incompressible method incorporates both the local changes
%% to compressibility due to reaction/heat release, and the large-scale
%% changes due to the background stratification.  This was originally
%% derived for an ideal gas equation of state for atmospherical flows.
%% Allowing the background pressure, $p_0$ to vary (e.g.\ in hydrostatic
%% equilibrium), differentiating pressure along particle paths gives:
%% \begin{equation}
%% \nabla \cdot (p_0^{1/\gamma} \Ub) = H
%% \end{equation}
%% where $\gamma$ is the ratio of specific heats and $H$ is the source.

%% \maestro\ is based on this method, generalizing this constraint to an
%% arbitrary equation of state and allowing for the time-variation of the
%% base state.

%% \subsection{Alternate Energy Formulation}

%% Several authors~\cite{KP:2012,VLBWZ:2013} showed that with a slightly
%% different momentum equation, the low Mach number system can conserve
%% an energy (that is, a quantity that looks like the compressible
%% energy, but formed using the low Mach number quantities).  This change
%% manifests itself as either a change to the buoyancy term or by
%% changing $\nabla \pi$ to $\beta_0 \nabla (\pi/\beta_0)$.  Furthermore,
%% \cite{VLBWZ:2013} showed that the new formulation better captures the
%% vertical propagation of gravity waves.  As of
%% Dec.\ 2013, this new formulation is the default in \maestro, a low
%% Mach number astrophysics code available through the CCSE web site.

\section{Projection Methods 101}

Most combustion hydrodynamics codes
(e.g.\ S3D~\cite{s3d}) solve the
compressible Navier-Stokes equations, which can be written in the form:
\begin{equation}
\Ub_t + \nabla \cdot F(\Ub) = S
\end{equation}
where $\Ub$ is a vector of conserved quantities, $\Ub = (\rho, \rho u,
\rho E)$, with $\rho$ the density, $u$ the velocity, $E$ the total
energy per unit mass, and $S$ are source terms.  This system
can be expressed as a coupled set of advection, diffusion, reaction
equations:
\begin{equation}
{\bf q}_t + A({\bf q}) \nabla {\bf q} + D = {\cal S}
\end{equation}
where ${\bf q}$ are called the primitive variables, $A$ is the advective
flux Jacobian, $A \equiv \partial F / \partial U$, $D$ are diffusion terms, 
and ${\cal S}$ are the transformed sources.  The eigenvalues of the
matrix $A$ are the characteristic speeds---the real-valued speeds at which
information propagates in the system, $u$ and $u
\pm c$, where $c$ is the sound speed.  Solution methods for the
compressible equations that are strictly conservative make use of this wave-nature to compute advective fluxes
at the interfaces of grid cells.  Diffusive fluxes can be computed 
either implicit or explicit in time, and are added to the advective fluxes,
and used, along with the source terms to update the state in time.  An
excellent introduction to these methods is provided by LeVeque's book
\cite{leveque}.  The timestep for these methods is limited by all three processes
and their numerical implementation.  Typically, advection terms are treated
time-explicitly, and the time step will be constrained by the time
it takes for the maximum characteristic speed to traverse one grid cell.
However, in low speed flow applications, it can be shown the acoustics 
transport very little energy in the system.  As a result, the time-step 
restrictions arising from numerical treatement of the advection terms
can be unnecessarily limited, even if A-stable methods are used to incorporate
the diffusion and source terms.

In contrast, solving low Mach number systems (including the equations of
incompressible hydrodynamics) typically involves a stage where one or more
advection-like equations are solved (representing, e.g.\ conservation of mass and
momentum), and coupling that advance with a divergence constraint on the velocity field.
For example, the equations of invicid constant-density incompressible flow
are:
\begin{eqnarray}
\Ub_t &=& -\Ub \cdot \nabla \Ub + \nabla p \label{eq:incompressible_u} \\
\nabla \cdot \Ub &=& 0
\end{eqnarray}
Here, $\Ub$ represents the velocity vector%
%
\footnote{Here we see an unfortunate conflict
of notation between the compressible hydro community and the
incompressible community.  In papers on compressible hydrodynamics,
$\Ub$ will usually mean the vector of conserved quantities.  In 
incompressible / low speed papers, $\Ub$ will mean the velocity vector.}
%
and $p$ is the dynamical pressure.  The time-evolution equation for
the velocity (Eq.~\ref{eq:incompressible_u}) can be solved using
techniques similar to those developed for compressible hydrodynamics,
updating the old velocity, $\Ub^n$, to the new time-level, $\Ub^\star$.
Here the `$^\star$' indicates that the updated velocity does not, in
general, satisfy the divergence constraint.  A projection method will
take this updated velocity and force it to obey the constraint
equation.  The basic idea follows from the fact that any vector
field can be expressed as the sum of a divergence-free quantity and
the gradient of a scalar.  For the velocity, we can write:
\begin{equation}
\Ub^\star = \Ub^d + \nabla \phi \label{eq:decomposition}
\end{equation}
where $\Ub^d$ is the divergence free portion of the velocity vector,
$\Ub^\star$, and $\phi$ is a scalar.  Taking the divergence of
Eq.~(\ref{eq:decomposition}), we have
\begin{equation}
\nabla^2 \phi = \nabla \cdot \Ub^\star
\end{equation}
(where we used $\nabla \cdot \Ub^d = 0$).
With appropriate boundary conditions, this Poisson equation can be
solved for $\phi$, and the final, divergence-free velocity can 
be computed as
\begin{equation}
\Ub^{n+1} = \Ub^\star - \nabla \phi
\end{equation}
Because soundwaves are filtered, the timestep constraint now depends only
on $|\Ub|$.

Extensions to variable-density incompressible
flows~\cite{bellMarcus:1992b} involve a slightly different
decomposition of the velocity field and, as a result, a slightly
different Poisson equation.  There is also a variety of different ways
to express what is being projected~\cite{almgren:bell:crutchfield},
and different discretizations of the divergence and gradient operators
lead to slightly different mathematical properties of the methods
(leading to ``approximate
projections''~\cite{almgrenBellSzymczak:1996}).  Finally, for
second-order methods, two projections are typically done per timestep.
The first (the `MAC' projection~\cite{bellColellaHowell:1991})
operates on the half-time, edge-centered advective velocities, making
sure that they satisfy the divergence constraint.  These advective
velocities are used to construct the fluxes through the interfaces to
advance the solution to the new time.  The second/final projection
operates on the cell-centered velocities at the new time, again
enforcing the divergence constraint.  The \pelelm\ algorithm performs
both of these projections.

The \pelelm\ algorithm builds upon these ideas, using a different
velocity constraint equation that captures the compressibility
due to local sources and large-scale stratification.

